\documentclass[conference]{IEEEtran}
%\documentclass[sigconf]{acmart}
\makeatletter
\def\ps@headings{%
\def\@oddhead{\mbox{}\scriptsize\rightmark \hfil \thepage}%
\def\@evenhead{\scriptsize\thepage \hfil \leftmark\mbox{}}%
\def\@oddfoot{}%
\def\@evenfoot{}}
\makeatother
\pagestyle{empty}
\usepackage{url}
\usepackage{graphicx,subfigure}
\usepackage{epstopdf}
\usepackage{amsmath}
\usepackage{algorithm}
\usepackage{algpseudocode}
\usepackage{amsmath}
\usepackage{amssymb}
\usepackage{amsthm}
\usepackage{epsfig}
\newtheorem{theorem}{Theorem}
\renewcommand{\algorithmicrequire}{\textbf{Input:}} % Use Input in the format of Algorithm
\renewcommand{\algorithmicensure}{\textbf{Output:}} % Use Output in the format of Algorithm
\usepackage{amsfonts}
%\newtheorem{theorem}{Theorem}[section]
\newtheorem{mydef}{Definition}[section]
%\newtheorem{lemma}{Lemma}[section]
\usepackage{multirow}
\usepackage{color}
\usepackage{array}
\usepackage{listings}
\usepackage{hyperref}
\usepackage[underline=true]{pgf-umlsd}
\newcommand{\tabincell}[2]
{\begin{tabular}
		{@{}#1@{}}#2\end{tabular}}
\usepackage{setspace}
\renewcommand{\labelitemi}{$\vcenter{\hbox{\tiny$\bullet$}}$}


\hyphenation{op-tical net-works semi-conduc-tor}




\begin{document}



\title{Replace with Your Project Title}

\author{\IEEEauthorblockN{1\textsuperscript{st} Given Name Surname}
\IEEEauthorblockA{\textit{dept. name of organization (of Aff.)} \\
\textit{name of organization (of Aff.)}\\
City, Country \\
email address}
\and
\IEEEauthorblockN{2\textsuperscript{nd} Given Name Surname}
\IEEEauthorblockA{\textit{dept. name of organization (of Aff.)} \\
\textit{name of organization (of Aff.)}\\
City, Country \\
email address}
\and
\IEEEauthorblockN{3\textsuperscript{rd} Given Name Surname}
\IEEEauthorblockA{\textit{dept. name of organization (of Aff.)} \\
\textit{name of organization (of Aff.)}\\
City, Country \\
email address}
}

\maketitle


\begin{abstract}
Phishing is an evident cybersecurity threat. Cybercriminals often pose as reputable organizations and send emails with links to phishing websites to unsuspecting individuals. Individuals who enter phishing websites, expose themselves to data breaches and malware. Companies spend time and money investing in cybersecurity training to protect their employees and customers from phishing attempts. Nevertheless, phishing is still prevalent in today’s society. In an October 2022 study, conducted by security provider SlashNext, found more than 255 million phishing attempts in email, mobile, and browser channels. SlashNext reports that there has been a 61% increase in phishing attempts since 2021. To solve this problem, we propose to create a phishing detection system based on machine learning algorithms. The problem is essentially a classification task of whether a link is legitimate or phishing. We will train the model to detect a phishing website based on a subset of features. 
\end{abstract}

\section{Introduction}
The section includes SEVERAL paragraphs summarizing your project. It is like the extended version of Abstract - you may use one paragraph for each of these parts - problem statement, dataset description, machine learning algorithms you will use to solve the problem, experimental results, and how your solutions are better as compared to existing solutions. Please try to limit Introduction to one page.

\section{Related Work}
This section summarizes existing solutions to the problem or similar problems. Please try to categorize these existing techniques and provide some discussion on the pros and cons of them. Don't forget to include references to any existing work you mention.  

\section{Our Solution}
This section elaborates your solution to the problem.

\subsection{Description of Dataset}
This subsection describes the dataset that you will use. In addition to the source of the dataset, it is also expected to include discussions on data proprocessing, e.g.,  statistical properties or visualization of the raw data, observation of some issues with the data (e.g., missing features, irrelevant features, outliers, etc.), some feature engineering to fix the problems (e.g., filling missing data, encoding of strings or characters into numerical values, normalization, and other operations as appropriate).

\subsection{Machine Learning Algorithms}
This subsection describes machine learning algorithms that you plan to use. For each ML algorithm, briefly 1) explain why it might be appropriate for the problem and 2) describe your main design. For example, if it is neural network, provide the network structure and your initial choice of some key parameters (e.g., activation function to use, number of layers, number of hidden nodes of each layer). You may change the parameters during the training process.  

\subsection{Implementation Details}
This subsection describes details of your implementation. Please focus on how you test and validate the performance, tune the hyperparameters, and select the best-performing models. Elaborate on techniques that you apply to improve the performance and explain why you use these techniques. You include few most important results/figures to illustrate your idea but do not let figures/tables dominate the content of the report. You can include few lines of critical code if needed. But please avoid paste lengthy code in your report. Please make sure the figures/tables/code snapshots are of appropriate size including the font size.

\section{Comparison}  
This section includes the following: 1) comparing the performance of different machine learning algorithms that you used, and 2) comparing the performance of your algorithms with existing solutions if any. Please provide insights to reason about why this algorithm is better/worse than another one.

\section{Future Directions}
This section lays out some potential directions for further improving the performance. You can image what you may do if you were given extra 3-6 months.

\section{Conclusion}
This section summarizes this project, i.e., by the extensive experiments and analysis, do you think the problem is solved well? which algorithm(s) might be better suitable for this problem? Which technique(s) may help further improve the performance? \\

Last but not the least, don't forget to include references to any work you mentioned in the report.
  

\bibliographystyle{IEEEtran}
\bibliography{}


\end{document}


